\documentclass{article}
\usepackage[utf8]{inputenc}
\usepackage{amsmath}
\usepackage{float}
\usepackage{amsfonts}
\usepackage{enumitem}

\title{Computer Vision\\Exercise 4}
\author{Gruppe 26\\Laszlo Korte\\Alexander Remmes-Weitz
}
\date{December 2022}

\begin{document}

\maketitle

\section*{Task 1}


\begin{enumerate}[label=(\alph*)]
    \item The dominant local direction of the keypoint is 45° because the histogram shows the highest peak at 0.3 for the 45° bin.
    \item The magnitude of the second largest bin (225°) is within the 80\% threshold of the highest peak (0.25/0.30 = 83\%). But the next lower bin (135°) is not (0.20/0.30 = 66\%). 
    
    That is why two keypoints are created: one for the dominant direction of 45° and one for the second strongest direction of 225°.
\end{enumerate}


\section*{Task 2}

\begin{align*}
\mathrm{UE}(G_1) &= \frac{\mathrm{Area}(S_1) + \mathrm{Area}(S_4) - \mathrm{Area}(G_1)}{\mathrm{Area}(G_1)} = \frac{3 + 8 - 4}{4} = 2 \\
\mathrm{UE}(G_2) &= \frac{\mathrm{Area}(S_3) - \mathrm{Area}(G_2)}{\mathrm{Area}(G_2)} = \frac{2-2}{2} = 0
\end{align*}

\end{document}
